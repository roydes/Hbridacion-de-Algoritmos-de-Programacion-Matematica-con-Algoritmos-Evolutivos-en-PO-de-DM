\chapter{Conclusiones y trabajos futuros} \label{Chapter8}
En el presente trabajo de investigación se propuso un nuevo enfoque de hibridación que tiene como objetivo aumentar la sinergia entre los buscadores local y global así como garantizar un mayor balance entre la operaciones de exploración y la explotación del espacio de búsqueda. El enfoque comprende siete lineamientos de diseño que pueden servir como marco de trabajo para futuros trabajos de hibridación. El lineamiento básico del enfoque plantea realizar una distribución de varias instancias de un buscador local en puntos aleatorios del espacio de búsqueda para garantizar la exploración. Sin embargo los operadores de los métodos de búsqueda local sólo explotan una vecindad relativamente pequeña del punto de inicio; por lo tanto se propone que un buscador global se encargue indicar a las diferentes instancias del buscador local, hacia dónde se encuentran las regiones más prometedoras. Estos lineamientos están motivados por las deficiencias encontradas en los trabajos de hibridación encontrados en la literatura especializada.

Teniendo en cuenta el enfoque de hibridación propuesto se procedió a realizar el diseño experimental de la propuesta de solución donde se seleccionó como método de programación matemática el método de Nelder-Mead y como buscador global el Algoritmo de Evaluación Diferencial. Los primeros experimentos consistieron en introducir operadores que incluían aleatoriedad en el procedimiento original del Nelder-Mead. Así, se diseñaron tres variantes que realizan expansiones de longitud aleatoria (NM-ELA) cuando ninguno de los operadores originales de reflexión, expansión y contracción mejoran al peor punto del simplex. Los resultados de las pruebas de Friedman y Bonferroni-Dunn indicaron un aumento significativo del desempeño del NM con los operadores ELA para los 5 problemas de diseño cinemático. 

Finalmente, se diseñaron cinco variantes iniciales que toman como algoritmo de búsqueda local al método Nelder-Mead. El procedimiento general consiste en dividir una población inicial generada aleatoriamente en $NS$ símpleces sobre los cuales cada instancia del Nelder Mead realizara una iteración por generación. El buscador global en este caso es el algoritmo de Evolución diferencial, el cual trabaja sobre el mejor punto de cada símplex. Los experimentos preliminares sobre estas variantes mostraron dificultades en los problemas de mayor dimensión. Por lo tanto, se diseñó la variante 6 que presenta una mayor capacidad de exploración y un mejor desempeño en forma general. Esto quedó evidenciado en los experimentos finales donde muestra un desempeño competitivo ante el algoritmo C-LSHADE y la ED/rand/1/bin.

Cada variante fue aplicada a seis problemas de optimización de diseño mecatrónico obteniendo resultados competitivos utilizando un número inferior de evaluaciones respecto a las variantes basadas en ED. Se obtuvieron nuevas soluciones que difieren en cuanto a forma y calidad de las ya reportadas en la literatura especializada. A continuación se describen las principales observaciones sobre el presente trabajo de investigación.
\section{Observaciones Finales}
\begin{enumerate}
	\item El enfoque propuesto permite diseñar algoritmos con mayor capacidad de exploración de diferentes regiones del espacio de búsqueda las cuales son localmente minimizadas por las instancias de los buscadores locales.
	\item A diferencia de los enfoques mayormente utilizados en el estado del arte, el buscador local tiene un papel preponderante en el proceso de minimización.
	\item El buscador global debe ser aplicado a un subconjunto élite de la población para garantizar mayor rapidez de convergencia.
	 \item Las variantes propuestas utilizan diferentes instancias del método Nelder-Mead modificado con un operador que concibe información global del espacio de búsqueda. La Evaluación Diferencial es utilizada como buscador global, actuando sobre los mejores puntos de cada símplex.
	 \item La variante HNMED-V6 aplica una estrategia de inicialización de los símpleces, ubicando los símpleces iniciales en la vecindad de los límites del espacio de búsqueda. Esta variante contempla la naturaleza geométrica del método Nelder-Mead, por lo que genera símpleces iniciales más regulares y de mayor tamaño. 
	 \item Se obtuvieron resultados competitivos en todos los problemas de optimización de diseño mecatrónico utilizando un número significativamente inferior a los ya reportados en la literatura para los seis problemas de optimización. En el caso de HNMED-V6 se utilizan menos símpleces iniciales ya que la los vértices iniciales se encuentran a mayor distancia por lo tanto son capaces de muestrear mayores áreas del espacio de búsqueda. Esto constituye una disminución de la complejidad temporal como de espacio en memoria.
	 \item Finalmente se obtuvieron nuevas soluciones para tres de los seis problemas de optimización. En el  caso MCE2 se encontró una nueva solución que describe un mecanismo diferente que alcanza una trayectoria con menor error respecto a los puntos de precisión. Para el caso 1 del gripper se obtuvieron vectores distantes que alcanzan el valor de $F(\vec{x})=0$, lo que demuestra la naturaleza multimodal de la función objetivo obtenida durante el diseño cinemático del mecanismo. Estos resultados demuestran la complejidad de estos problemas, sobre todo los de diseño cinemático. 
\end{enumerate}  
	 
	 \section{Objetivos cumplidos}
	  El objetivo general de la presente investigación fue diseñar un Algoritmo Híbrido de Programación Matemática con elementos de Algoritmos Evolutivos  para resolver problemas no lineales de optimización de  Diseño Mecatrónico. El cual fue cumplido, ya que bajo el enfoque de hibridación propuesto se generaron seis algoritmos competitivos que utilizan un número inferior de evaluaciones de la función objetivo respecto a los encontrados en la literatura especializada. Para cumplir el objetivo general se satisficieron los siguientes objetivos específicos:
	 \begin{enumerate}
	 	\item	Estudiar el comportamiento de al menos dos algoritmos de programación matemática al resolver problemas de diseño mecatrónico.
	 	\item	Determinar bondades y deficiencias algoritmos de programación matemática.
	 	\item	Diseñar un Algoritmo Híbrido de programación matemática con elementos de algoritmos evolutivos para mejorar la búsqueda.
	 	\item	Aplicar el Algoritmo Híbrido a problemas de optimización de diseño mecatrónico.
	 	\item	Realizar pruebas de medición de desempeño mediante el análisis estadístico y medición del número de evaluaciones del algoritmo propuesto.
	 	\item	Comparar los resultados obtenidos con el desempeño de los algoritmos identificados en la literatura especializada.
	 \end{enumerate}
	 
 \section{Contribuciones}
 Las contribuciones del presente trabajo se dividen en dos áreas fundamentales. La primera es proveer a la comunidad científica de un nuevo enfoque de diseño para la hibridación entre métodos de búsqueda local y global, que permite la implementación de algoritmos eficientes para problemas de optimización no lineales con restricciones. Por otra parte, los algoritmos propuestos permiten una optimización eficaz y eficiente de un conjunto diverso de problemas mecatrónicos complejos; por lo que quedan a disposición de los investigadores dedicados al diseño mecatrónico para futuros problemas de este tipo. 
 \section{Trabajo Futuro}
 De acuerdo a los resultados obtenidos durante la investigación se contemplan los siguientes trabajos futuros:
 \begin{enumerate}
    \item Aplicar el enfoque de hibridación en el diseño de algoritmos que combinen otros métodos de búsqueda local y con metaheurísticas para la búsqueda global. 
    \item Continuar investigando el presente conjunto de problemas de optimización de diseño mecatrónico para alcanzar mejores resultados en los problemas MCE2,GCE2 y SCE1.
    \item Aplicar el enfoque de hibridación para la resolución de problemas de optimización no lineales con restricciones que permitan una validación general del mismo. 
    \item Aplicar las variantes propuestas a nuevos problemas de optimización no lineales de diseño mecatrónico.
    \item Comprobar el desempeño de las variantes con otros mecanismos para el manejo de restricciones y reglas de acotamiento para las variables de diseño.
    \item Realizar un análisis detallado que incluya estadísticas sobre las operaciones realizadas por cada buscador durante la minimización como  conteo de aplicación de operadores, mejora de la función objetivo por operador, evaluación de la diversidad de la población entre otras.
    \item Realizar un estudio sobre los parámetros del algoritmo HNMED-V6.
 
 \end{enumerate}
 