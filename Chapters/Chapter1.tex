% Chapter 1
\chapter{Introducción}% Main chapter title

\label{Chapter1} % For referencing the chapter elsewhere, use \ref{Chapter1} 

%----------------------------------------------------------------------------------------

% Define some commands to keep the formatting separated from the content 
\newcommand{\keyword}[1]{\textbf{#1}}
\newcommand{\tabhead}[1]{\textbf{#1}}
\newcommand{\code}[1]{\texttt{#1}}
\newcommand{\file}[1]{\texttt{\bfseries#1}}
\newcommand{\option}[1]{\texttt{\itshape#1}}

%----------------------------------------------------------------------------------------
   
El desarrollo moderno industrial ha generado la necesidad de aumentar tanto la eficiencia, como la eficacia de los procesos de producción. En pos de lograr este objetivo las diferentes maquinarias y dispositivos industriales comenzaron a automatizarse para lograr sistemas de producción cada vez más ágiles, precisos y controlables. Al aumentar la complejidad de los procesos industriales se comenzaron a desarrollar además, maquinarias inteligentes, capaces de actuar efectivamente ante determinadas situaciones del proceso productivo, en lugar operar de forma programática. 

La Mecatrónica surge entonces como una rama de la ingeniería capaz de lograr una integración sinérgica entre la ingeniería mecánica, la electrónica, el control  automático y el cómputo inteligente \cite{silva_mechatronics:_2010}. Diferentes fuentes coinciden que el término Mecatrónica se utilizó por primera vez en Japón en la década de 1960 por la empresa Yaskawa, en documentos de aplicación de marca. Con el paso del tiempo, los avances en la Mecatrónica comenzaron a ser utilizados en las máquinas expendedoras, cámaras de enfoque automático y puertas automáticas. Con el advenimiento de las nuevas Tecnologías de la Información en la década de 1980, los microprocesadores se introdujeron en los sistemas mecatrónicos, mejorando el rendimiento significativamente. En la década de 1990, los avances en inteligencia computacional se aplicaron a la mecatrónica en formas que revolucionaron el campo \cite{bishop_mechatronics_2002}. Actualmente los sistemas mecatrónicos se utilizan en una amplia gama de la industria, donde se destacan: robots industriales, automóviles modernos, aeronaves y vehículos espaciales, sistemas de ventilación y electrodomésticos inteligentes. 

Un dispositivo mecatrónico es un conjunto complejo de componentes que abarca tecnologías de diferentes áreas de la ingeniería. De forma general, se pueden identificar cuatro tipos de componentes o subsistemas principales que lo integran: Sistemas Eléctricos, Sistemas Mecánicos, Sistemas Computacionales y Sistemas de Información.  Según  Shetty y Kolk \cite{shetty_mechatronics_2010} el estudio de los sistemas mecatrónicos se puede dividir en las siguientes áreas del conocimiento:
	
	\begin{itemize}
	\item[1.]\textbf{Modelado de Sistemas Físicos}: Incluye el diseño óptimo de Sistemas Mecánicos y Eléctricos. El diseño de sistemas mecánicos se ocupa del comportamiento de la materia bajo la acción de fuerzas. Tales sistemas se categorizan como rígidos, deformables o fluidos. La cinemática y mecánica newtoniana proporcionan la base para la mayoría de los sistemas mecánicos y consta de tres conceptos independientes y absolutos: el espacio, el tiempo y la masa. Un cuarto concepto, la fuerza, también está presente, pero no es independiente de los otros tres. Los sistemas eléctricos en los sistemas mecatrónicos requieren una comprensión del análisis de circuitos de corriente continua (CC) y de corriente alterna (CA), incluyendo impedancia, potencia y dispositivos electromagnéticos así como semiconductores (tales como diodos y transistores). También se incluye el estudio de la energía en varias formas incluyendo potencial, cinética, eléctrica, calor, química, nuclear y radiante.
	\item[2.]\textbf{Sensores y actuadores}: Abarca el estudio de sensores lineales y rotacionales, sensores de aceleración, fuerza, torque, presión, temperatura, flujo, sistemas de detección de luz, imágenes y visión. Dispositivos de fibra óptica, micro y nano-sensores. Incluye el estudio de actuadores electromecánicos, motores, actuadores hidráulicos y neumáticos, actuadores piezoeléctricos, micro y nano-actuadores.
	\item[3.]\textbf{Señales y Sistemas}: Abarca el modelado mecatrónico y la respuesta de sistemas dinámicos. Estudia la estabilidad, capacidad de control y monitoreo de los sistemas, el diseño de control óptimo, el diseño de control adaptativo y no-linear, el control inteligente, las redes neuronales y sistemas de lógica difusa.
	\item[4.]\textbf{Computadoras y Sistemas Lógicos}: Incluye el diseño de sistemas lógicos y sistemas de comunicación, la lógica secuencial sincrónica y asincrónica, la arquitectura de computadoras y microprocesadores, interfaces, controladores lógicos programables y computadoras de control integradas.
	\item[5.]\textbf{Adquisición de datos y software}: Estudia los sistemas de adquisición de datos, transductores y sistemas de medición, sistemas amplificadores y acondicionadores de señales, sistemas de instrumentación basado en computadoras, ingeniería de software y grabación de datos.
	
\end{itemize}

 Teniendo en cuenta estos elementos, Shetty y Kolk \cite{shetty_mechatronics_2010}, definen el proceso de diseño de un sistema mecatrónico según se muestra en la Figura \ref{DMecatronico}. Se puede observar que una tarea continua durante todo este proceso es la optimización del diseño, la cual resulta una condición necesaria para que el producto final se desempeñe eficientemente y así lograr la competitividad requerida por la industria. La optimización presenta mayor relevancia en el proceso de Modelado y Simulación. El modelado es el proceso  donde se representa el comportamiento de un sistema real mediante un conjunto de  ecuaciones matemáticas. El término sistema real es sinónimo de sistema físico, un sistema cuyo comportamiento se basa en la materia y la energía. Los modelos son estructuras de causa y efecto, aceptan información externa y la procesan con su lógica y ecuaciones para producir una o más salidas \cite{fritzson_basic_2011}.
	
	\begin{figure}[htb]
		\begin{center}
		 \resizebox {\textwidth} {!} {	
			\includegraphics[width=\linewidth, height=10cm]{Figures/ProcesoDM}
		}
			\caption[Proceso de Diseño Mecatrónico]{Proceso de Diseño Mecatrónico (Tomado de \cite{shetty_mechatronics_2010}).}
			\label{DMecatronico}
		\end{center}
	\end{figure}


 El objetivo de la optimización es establecer una configuración óptima del sistema. Las propiedades de un sistema electromecánico pueden ser descritas matemáticamente mediante magnitudes físicas. Los valores de estas magnitudes, y por tanto, el estado del sistema pueden ser descritos mediante funciones matemáticas en correspondencia con el tipo de maquinaria y su descripción matemática.  Estas funciones matemáticas constituyen las funciones objetivo de los problemas de optimización en el diseño mecatrónico. Por la complejidad que presentan estos sistemas es común que se describan problemas de optimización no lineales con  restricciones.

 \section{Descripción de la problemática}
  El presente trabajo investigativo se enfocará en la resolución de seis problemas de optimización. Primeramente se resolverán tres casos de estudio del problema de la ``Síntesis Óptima de un Mecanismo de Cuatro Barras''. Los problemas siguientes serán dos casos de estudio de la ``Síntesis Óptima de un Efector Final de Tres Dedos''. Por último, se abordará el problema de `Optimización del costo de la generación de energía en una Microrred Eléctrica No Interconectable''. Estos problemas son objeto de estudio en el Centro de Innovación y Desarrollo Tecnológico  en Cómputo (CIDETEC) del Instituto Politécnico Nacional en el área de Mecatrónica. Los trabajos de investigación en los que han sido abordados servirán para validar la competitividad de la solución propuesta. 

 Las características de los problemas a resolver se encuentran resumidas en la Tabla \ref{Resumen de características de los problemas de optimización de Diseño Mecatrónico.}, donde  $n$, es el número de variables del problema, $LI$ es el número de restricciones de desigualdad lineal, $NI$ es el número de restricciones de desigualdad no lineales,  $LE$ es el número de restricciones de igualdad lineal y $NE$ las restricciones de igualdad no lineales . Además, se describe la nomenclatura a utilizar para los casos de estudio. El primer caracter de la nomenclatura corresponde al problema general, donde se adopta la siguiente convención: \textit{M} para el mecanismo de cuatro barras, \textit{G} para el efector final o gripper, y \textit{S} para la micro red o smart grid. Los carateres restantes serán CE<X> correspondiente al Caso de Estudio X. Estos problemas de optimización han sido abordados en diferentes investigaciones, tanto por medio de algoritmos de programación matemática, como por algoritmos Bio-inspirados.

\begin{table}
	\centering
	\caption{Resumen de características de los problemas de optimización de Diseño Mecatrónico.}
	\label{Resumen de características de los problemas de optimización de Diseño Mecatrónico.}
	\resizebox{\textwidth}{!}{
		\begin{tabular}{|p{6cm}|l|l|l|l|l|l|l|} 
			\hline
			Problema~                                                          & Caso de estudio & Tipo de Función & n  & LI & NI & LE & NE  \\ 
			\hline
			\multirow{3}{=}{Síntesis Óptima de un Mecanismo de Cuatro Barras}  & MCE1            & No lineal       & 15 & 9  & 0  & 0  & 0   \\ 
			\cline{2-8}
			& MCE2            & No lineal       & 6  & 4  & 0  & 0  & 0   \\ 
			\cline{2-8}
			& MCE3            & No lineal       & 19 & 13 & 0  & 0  & 0   \\ 
			\hline
			\multirow{2}{=}{Síntesis Óptima de un Efector Final de Tres Dedos} & GCE1            & No lineal       & 14 & 8  & 0  & 0  & 0   \\ 
			\cline{2-8}
			& GCE2            & No lineal       & 14 & 8  & 0  & 0  & 0   \\ 
			\hline
		Optimización del costo de la generación de energía en una Microrred Eléctrica No Interconectable   & SCE1            & Cuadrática      & 4  & 3  & 0  & 1  & 0   \\
			\hline
		\end{tabular}
	}
\end{table}


En el caso del  diseño del Mecanismo de Cuatro Barras se destacan los trabajos realizados por Portilla y Mezura, quienes han aplicado métodos aproximados para la resolución del problema obteniendo resultados positivos. 

 En \cite{portilla-flores_dynamic_nodate} se plantea el diseño del Mecanismo de Cuatro Barras como un problema de optimización dinámica mono-objetivo (PODMO) que considera los modelos cinemático y dinámico de todo el sistema, así como un conjunto de restricciones incluyendo una restricción dinámica para lograr la síntesis de un mecanismo de cuatro barras que proporciona un movimiento simétrico en su enlace de balancín. En el artículo se describe la aplicación del Algoritmo de Colonia de Abejas Modificado donde el operador de variación utilizado por las abejas empleadas y observadoras para generar una nueva solución, incluye un mecanismo de recombinación. A partir del análisis de la simulación y los resultados obtenidos, se observó que las soluciones encontradas por el algoritmo propuesto conducen a un diseño más adecuado basado en el enfoque dinámico.
% \usepackage{multirow}


En el artículo ``Síntesis Óptima de un Mecanismo de Cuatro Barras utilizando el algoritmo de Forrajeo de Bacterias Modificado'', algoritmo de Forrajeo de Bacterias (BFOA por sus siglas en inglés) es adaptado para resolver el problema de optimización mediante la adición de una técnica de manipulación de restricciones capaz de incorporar un criterio de selección para los dos objetivos establecidos en el análisis cinemático del problema. Se diseñó un mecanismo de diversidad para favorecer la exploración del espacio de búsqueda y los resultados son comparados con los proporcionados por cuatro algoritmos encontrados en la literatura especializada utilizada para resolver problemas de diseño mecánico\cite{mezura-montes_optimum_2014}. Del mismo modo, en \cite{herne1_two_swim_2016} se aplica un algoritmo basado en el Forrajeo de Bacterias que utiliza dos operadores llamados Nados (TS-MBFOA por sus siglas en inglés). El algoritmo utiliza además, un mecanismo de sesgo para el conjunto inicial de bacterias, un operador de búsqueda local de segundo orden y un uso limitado de la etapa de reproducción.
  
La Evolución Diferencial (ED) ha presentado el mejor desempeño entre los Algoritmos Bio-inspirados que han sido aplicados a estos problemas. Lo que se evidencia en publicaciones como ``Evolución Diferencial para el Ajuste Óptimo de la Ganancia de control de un Mecanismo de Cuatro Barras'' donde se minimiza la variación del error de velocidad de un mecanismo de cuatro barras con fuerzas de resorte y de amortiguación utilizando la ED con un mecanismo de manejo de restricciones. Todas las ejecuciones del algoritmo convergieron al vector de la variables de diseño óptimo y la solución encontrada puede ser considerada como la global. Por otro lado, la media reportada de los tiempos de ejecución  del algoritmo fue de diez minutos. Igualmente, la ED ha sido aplicada con éxito en problemas de optimización mono y multi-objetivo con restricciones de diseños mecatrónicos \cite{portilla-flores_integration_2007} \cite{villarreal-cervantes_control_2013}.
  
  De forma similar la ED ha obtenido resultados importantes en la resolución del problema de la ``Síntesis Óptima de un Efector Final de Tres Dedos''. En \cite{mezura-montes_dynamic_2015} se plantea la configuración de parámetros para un algoritmo de Evolución Diferencial con Variantes Combinadas (DECV por sus siglas en inglés) para de mejorar su desempeño. Se resuelven los dos casos de estudio del Efector Final. Se realiza una calibración de los parámetros principales del algoritmo, para posteriormente agregar un mecanismo de control para el parámetros asociado a la mutación. Los resultados sugieren, que la aplicación del mecanismo de control de parámetros permite al algoritmo base DECV alcanzar resultados altamente competitivos con un costo computacional menor, medido por el número de evaluaciones de soluciones potenciales.
  
  

  La optimización del despacho económico de una micro-red en modo aislado es abordada en \cite{ramabhotla2014economic} utilizando el Método del Gradiente Reducido.  Los algoritmos de gradiente reducido evitan el uso de parámetros de penalización de restricciones. En cambio, la búsqueda se realiza a lo largo de curvas que permanecen cerca del conjunto factible. Las restricciones de igualdad son utilizadas para eliminar un subconjunto de las variables, reduciendo así el problema original a un problema restringido en el espacio de las variables restantes. La optimización se obtiene minimizando la función de costo del sistema mientras que satisface la demanda de carga. El costo de operación, mantenimiento y los costos de inversión se consideran en las funciones de costos de los micro-recursos.  El costo total mínimo de operación del sistema se obtiene al comparar diferentes escenarios de los micro-recursos en la micro-red. 
  
 
 En \cite{heredia-ramirez_optimal_2014} se analiza el Flujo Óptimo de Potencia (FOP) de una MR, el cual consiste en resolver ecuaciones que caracterizan un sistema eléctrico (potencia activa y reactiva de cada nodo) ajustando los valores de las variables de control (voltajes o potencias) para optimizar un parámetro específico del sistema, representado por una función objetivo. Para resolver el problema se utiliza un algoritmo matemático basado en el Método de Gradiente. El gradiente permite medir la sensibilidad de la función objetivo con respecto a los cambios en las variables de control. Originándose en una dirección opuesta al gradiente, se alcanza un punto mínimo factible con un valor de función inferior. La repetición de este proceso conduce a la solución óptima del sistema.
 
Finalmente en \cite{zapata_zapata_control_2017}  se proponen dos variantes de la Evolución Diferencial: C-LSHADE y  LSHADE-CV. Estas variantes están basadas en una variante competitiva de la Evolución llamada LSHADE, la cual incorpora un mecanismo de control que emplea una memoria histórica de parámetros exitosos para la configuración del factor de escala ($F$), la probabilidad de cruzamiento ($CR$) y una reducción lineal de la población.  C-LSHADE agrega las adaptaciones necesarias para solucionar problemas de optimización con restricciones. Por otra parte, LSHADE-CV aplica la estrategia de LSHADE al algoritmo DECV. En este trabajo se resuelven de manera satisfactoria los problemas de diseño cinemático que serán objeto de estudio en el presente trabajo. Además, se resuelve el problema del Smart Grid donde se minimizó el costo del suministro de energía en cada hora del día. La solución encontrada aumentó la generación de energía eléctrica mediante las utilización de las Fuentes de Energía Renovable maximizando el ahorro. Todos los problemas fueron resueltos utilizando un número inferior de evaluaciones de la función objetivo respecto a la ED/rand/1/bin. 
  
  Como se puede observar, los problemas de diseño a resolver han sido atacados por los dos grupos más utilizados en problemas de optimización de diseño mecatrónico: los métodos de programación matemática y los algoritmos aproximados, entre los que se encuentran los algoritmos bio-inspirados. Los métodos de programación matemática son utilizados para encontrar mínimos locales en funciones objetivo de una o varias variables y pueden ser clasificados de acuerdo al orden de la derivada utilizada en la aplicación del método \cite{bishop_mechatronics_2002}:
  \begin{enumerate}
  	\item Métodos de Orden Cero (Comparativos)
  	\begin{enumerate}
  		\item Métodos Comparación coordinada
  		\item Métodos basados en simplex
  		\item Métodos estocásticos  
  	\end{enumerate}
  	\item Métodos de Primer Orden (Gradiente y Cuasi-gradiente)
  	\begin{enumerate}
  		\item Métodos de direcciones asociadas
  		\item Métodos variable-métrica 
  	\end{enumerate}
  	\item Métodos de Segundo Orden (Método de Newton).
  \end{enumerate}
  Una deficiencia sustancial de los métodos clásicos radica en que al optimizar funciones con gran cantidad de mínimos locales generalmente convergen en un mínimo cercano al punto de inicio perdiendo el mínimo global. También una característica de estos métodos, es su sensibilidad al punto de origen para encontrar la solución. Por otra parte, los algoritmos que utilizan la derivada de la función para obtener los puntos críticos, asumen como es lógico que que la función sea derivable en el espacio de búsqueda del problema. Estas características pueden restringir su aplicación, debido a que en problemas reales es común encontrar funciones que no son continuas y ni diferenciables en todo el espacio de búsqueda. A pesar de estas restricciones, los métodos de programación matemática son considerablemente menos costosos computacionalmente, presentando mayor rapidez de convergencia hacia los óptimos de la función.
    
  Por lo anterior, nuevos enfoques han sido aplicados para la resolución problemas de optimización global, siendo los algoritmos bio-inspirados, en particular los tipo evolutivo los que alcanzan mejores resultados. Los algoritmos bio-inspirados emulan un fenómeno existente en la naturaleza mediante el uso de una metáfora biológica del comportamiento de cierto proceso natural o agente biológico. Estos algoritmos son métodos de resolución de problemas que tienen diversas  aplicaciones, particularmente en la resolución de problemas de optimización global como es el caso del Diseño Mecatrónico \cite{yang_swarm_2013}. En general los algoritmos bio-inspirados generan soluciones iniciales aleatoriamente, lo que les permite buscar de forma más eficiente en el espacio de todas las posibles soluciones. Entre los algoritmos más utilizados se pueden encontrar los siguientes:

  \begin{enumerate}
  	\item Algoritmos estocásticos 
  	\item Algoritmos de Recocido Simulado 
  	\item Algoritmos Evolutivos 
  	\item Algoritmos de Inteligencia de Colectiva
  \end{enumerate}

Una estrategia efectiva es combinar dos algoritmos para obtener un nuevo procedimiento que incorpore las mejores características de cada enfoque. A este enfoque se conoce como hibridación y será abordado con mayor profundidad en el Capítulo \ref{Chapter6}. Dentro de la clasificación de algoritmos híbridos se encuentran los Algoritmos Meméticos, este enfoque se basa en la interacción de los procesos de búsqueda global y local. Estos algoritmos son metaheurísticas basadas en población. Básicamente están constituidos por un algoritmo bio-inspirado para la búsqueda global y un conjunto de algoritmos de búsqueda local que se activan dentro del ciclo de generación del buscador global \cite{ferrante_handbook_2012}. Actualmente, la aplicación de estos algoritmos se ha incrementado debido a que muestran un mejor rendimiento en espacios de búsqueda complejos. 

En este ámbito es necesario destacar el trabajo 
realizado en \cite{VegaMEC1} donde se propone un algoritmo memético que utiliza como buscador global el algoritmo Modificado de Colmena de Abejas, con la adición del método de Caminata Aleatoria. Los resultados de la simulación muestran un control de alta precisión de la trayectoria propuesta para el mecanismo de Cuatro Barras diseñado.

La principal deficiencia de los algoritmos evolutivos y sus híbridos se encuentra en que, debido a la utilización de una población y su naturaleza estocástica necesitan un número elevado de evaluaciones de la función objetivo durante la optimización. Por tanto, son considerablemente más lentos que sus contrapartes de Programación Matemática cuando se aplican a espacios de búsquedas complejos. 
\section{Planteamiento del Problema de investigación}
  
Con base en los argumentos presentados en la sección anterior, se puede observar que la problemática existente entorno el objeto de estudio (problemas a resolver), es que, debido a la complejidad que presentan estos problemas, los algoritmos empleados hasta el momento necesitan un número elevado de evaluaciones para obtener soluciones de calidad. Teniendo en cuenta lo antes planteado se formula el siguiente problema de investigación:

  \textit{¿Cómo resolver problemas de optimización no lineales de Diseño Mecatrónico disminuyendo el costo computacional, medido por el número de evaluaciones de soluciones potenciales, para obtener una solución competitiva?}
  \section{Objetivo general}
  Para resolver el problema planteado se define como objetivo general:
  
  
 \textit{ Diseñar un Algoritmo Híbrido de Programación Matemática con elementos de Algoritmos Evolutivos  para resolver problemas no lineales de optimización de  Diseño Mecatrónico.} 
  \subsection{Objetivos específicos}
  El objetivo general se desglosa en los siguientes objetivos específicos:
  \begin{enumerate}
  	\item	Estudiar el comportamiento un algoritmo de programación matemática al resolver problemas de optimización de diseño mecatrónico.
  	\item	Determinar bondades y deficiencias del método de Programación Matemática en la resolución de problemas de optimización diseño mecatrónico.
  	\item	Diseñar un algoritmo híbrido basado en un método de programación matemática combinado con un algoritmo evolutivo para mejorar la búsqueda.
  	\item	Aplicar el algoritmo híbrido a problemas de optimización de diseño mecatrónico.
  	\item	Realizar pruebas de medición de desempeño mediante el análisis estadístico y medición del número de evaluaciones del algoritmo propuesto.
  	\item	Comparar los resultados obtenidos con el desempeño de los algoritmos identificados en la literatura especializada.
  \end{enumerate}
  
  \section{Hipótesis de investigación}
  La propuesta de solución se basa en la siguiente hipótesis de la investigación que plantea:\\
\textit{  Un Algoritmo Híbrido basado en un método de programación matemática y enriquecido con operadores de algoritmos evolutivos realizará menos evaluaciones, lo que reduciría el tiempo para obtener soluciones iguales o mejores a las reportadas en la literatura especializada de problemas de optimización de Diseño Mecatrónico.}
  
  \section{Justificación}
  
  La Mecatrónica es un área en constante desarrollo, cada año se generan nuevas necesidades en la industria que contribuyen a la evolución de las tecnologías que son aplicadas para la creación de los diferentes productos. Las áreas de acción de la mecatrónica son diversas. Además del escenario más frecuente enfocado a la manufactura, se pueden encontrar aplicaciones en robots de servicio  \cite{munaro2014fast} \cite{jiang2015novel}, sistemas de navegación marítima \cite{shi2017advanced}  y aplicaciones en el campo de la medicina \cite{burgner2015continuum}. Los algoritmos aplicados para solucionar problemas optimización de diseño mecatrónico son un elemento clave dentro del software utilizado en proyectos de mecatrónica. Además del enfoque cíclico del proceso de diseño, las fases que lo integran ocurren secuencialmente. Por tanto, el enfoque tradicional en mecatrónica es un enfoque de ingeniería secuencial. Según \cite{shetty_mechatronics_2010}, una encuesta de Standish Group sobre proyectos de mecatrónica dependientes de softwares plantea que: los proyectos terminados presentan un 222\% de rebasamiento del tiempo planificado inicialmente y sólo el 16.2\% de todos los proyectos se completan a tiempo y dentro del presupuesto. Por esta razón, algoritmos de optimización más rápidos pueden ayudar a reducir los tiempos en el proceso de diseño. Debe tenerse en cuenta que el diseño de un sistema mecatrónico puede incluir diferentes componentes o subsistemas, que deben ser modelados, optimizados y simulados en las fases correspondientes.
  
  Por otra parte, existe un número creciente de sistemas mecatrónicos que son controlados en tiempo real por computadoras, cuyo desempeño puede verse mejorado. Se destacan en este tipo de sistemas, tareas como la planificación de movimiento en tiempo real de vehículos autónomos \cite{frazzoli_real-time_2002},  la optimización de la fuerza de agarre en tiempo real  de mecanismos manipuladores de múltiples dedos \cite{liu_real-time_2004}, y la optimización de recursos en Smarts Grids \cite{kumar2017smart} \cite{ahat2013smart} \cite{mortaji2016smart}. Para estos sistemas la rapidez del algoritmo en encontrar soluciones de calidad es de suma importancia.
  
  En los casos de estudio a resolver, se han aplicado tres enfoques: los algoritmos bio-inspirados, los métodos de programación matemática y los algoritmos meméticos; lo que deja espacio a la investigación y análisis de nuevos enfoques de hibridación de algoritmos aplicados al objeto de estudio del presente trabajo. Además, se debe tener en cuenta que estos son problemas de optimización global, esto implica que los mínimos encontrados hasta el momento no son necesariamente los óptimos globales. Teniendo en cuenta la diversidad de algoritmos efectivos que se puede encontrar en la literatura especializada, la hibridación supone una vía factible para reducir los tiempos de ejecución al tiempo que se garantiza la convergencia a soluciones competitivas en problemas optimización de diseño ingenieril \cite{sacco_metropolis_2008} \cite{koch_hybridization_2009}  \cite{jovanovic_cuckoo_2016}.
  


  \section{Alcances}
  \begin{itemize}
  	\item[-] El algoritmo propuesto estará destinado a resolver únicamente casos de estudios de los problemas generales planteados en el Capítulo \ref{Chapter5}.
  	\item[-] Los resultados alcanzados por el algoritmo propuesto deberán ser competitivos en el contexto de los problemas a resolver.
  	\item [-] Los resultados alcanzados por la presente investigación serán sometidos a pruebas de desempeño utilizando métricas encontradas en la literatura especializada.
  	
  \end{itemize}
  
  \section{Limitaciones}
  \begin{itemize}
  	\item[-] La investigación se limita a la optimización del diseño y la simulación de las soluciones obtenidas. No se implementarán físicamente los prototipos.
  	\item[-] Se asumirá pérdida de propiedades de los métodos clásicos al incorporar mecanismos heurísticos en ellos. 
  	
  \end{itemize}

 \section{Organización del documento}
A continuación se describe la estructura y contenido de las secciones que conforman este documento:
  \begin{itemize}
	\item[1.]\textbf{Capítulo 2. Optimización}. En este capítulo se introducen los conceptos de
optimización y convexidad de una función, se describen los componentes principales de un problema de
optimización, su clasificación y tipos de técnicas utilizadas para
resolverlos. 
\item[2.]\textbf{Capítulo 3. Programación Matemática}. Se presentan los principales métodos de Programación Matemática. Se describen los métodos de acotamiento, métodos directos, así como los métodos basados en gradiente. Se seleccionan dos candidatos para realizar la hibridación con los algoritmos evolutivos.
	\item[3.]\textbf{Capítulo 4. Computación Evolutiva}. El capítulo consiste en una introducción a los conceptos básicos del Cómputo Evolutivo. Se describen los principales algoritmos en está área de la Inteligencia Artificial. Se presenta con mayor detalle el algoritmo de Evolución Diferencial.
	\item[4.]\textbf{Capítulo 5. Problemas de Optimización de Diseño Mecatrónicos}. Se describen los problemas de diseño mecatrónico como problemas de optimización mono-objetivos con restricciones. Se describen tres casos de la síntesis óptima del mecanismo de cuatro barras y dos casos de la síntesis Óptima de un Efector Final y la Optimización  de generación de energía en una
microrred aislada. 
	\item[5.]\textbf{Capítulo 6. Hibridación de Métodos de Programación Matemática con Algoritmos Evolutivos}. En este capítulo  se presenta la solución propuesta de la presente investigación. Se plantea un enfoque de hibridación que contiene diferentes lineamientos de diseño. Además se describen seis variantes de un algoritmo híbrido basado en el Método Nelder Mead, y que utiliza la Evolución Diferencial como buscador global. 
	\item[6.]\textbf{Capítulo 7. Experimentos y Resultados}. Este capítulo describe el diseño experimental mediante el cual se obtiene la propuesta de solución. Se aplican técnicas de estadística descriptiva e inferencial para el análisis del comportamiento de los algoritmos y la comparación con los métodos más competitivos encontrados en la literatura.
\item[7.] \textbf{Conclusiones y Trabajo futuro}. Se realizan observaciones finales sobre los resultados obtenidos en los problemas de optimización y consideraciones sobre el comportamiento de los algoritmos propuestos. Se describen las aportaciones más importantes de la investigación así como los trabajos futuros a realizar teniendo en cuenta los resultados alcanzados.

  \end{itemize}
\section{Conclusiones del capítulo}
  En el capítulo se fundamenta la presente investigación cuyo objeto de estudio es un conjunto de problemas no lineales de Optimización de Diseño Mecatrónico.  Hasta el momento los enfoques más exitoso han sido los basados en metaheurísticas y sus híbridos. Sin embargo, se evidencia la necesidad de continuar investigando los problemas de optimización descritos así como diseñar y aplicar nuevos enfoques que reduzcan el costo computacional, específicamente la complejidad temporal medida en el número de evaluaciones de la función objetivo. Para lograr esto se plantea como objetivo diseñar un algoritmo híbrido que utilice un método de Programación Matemática como base, formulando la hipótesis de que este esquema reduciría el número de evaluaciones necesarias para encontrar resultados competitivos. 